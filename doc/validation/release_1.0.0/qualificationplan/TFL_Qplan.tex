\documentclass{article}
\usepackage[parfill]{parskip}
\usepackage[absolute]{textpos}
\usepackage{graphicx}
\usepackage[T1]{fontenc}
\usepackage[scaled]{helvet}
%\usepackage[scaled]{helvet}

\usepackage{lscape}
\usepackage{courier}
\renewcommand*\familydefault{\sfdefault}
\usepackage[left=1in,top=1in,bottom=1.25in,right=1in]{geometry}
\usepackage{color}
\usepackage{fancyhdr,lastpage}
\pagestyle{fancy}
\rhead{\scriptsize \opnum:  \optitle}
%\chead{Middle top}
\newenvironment{changemargin}[2]{%
\begin{list}{}{%
\setlength{\topsep}{0pt}%
\setlength{\leftmargin}{#1}%
\setlength{\rightmargin}{#2}%
\setlength{\listparindent}{\parindent}%
\setlength{\itemindent}{\parindent}%
\setlength{\parsep}{\parskip}%
}%
\item[]}{\end{list}}
\lhead{\includegraphics[scale=.6]{logonew.pdf}\\\smallskip\scriptsize Qualification}

\fancyfoot[C]{Page \thepage\ of \pageref{LastPage}}
\usepackage[colorlinks=true,linkcolor=black,urlcolor=blue]{hyperref}

%\newcommand{\opnum}{Release 2} 
\newcommand{\optitle}{Pharmacometrics TFL Generator Qualification Plan}
\newcommand{\opnum}{Release 1.0.0} 
\newcommand{\tfl}{Pharmacometrics TFL Generator}

\begin{document}

\vspace*{3cm}
{\Large{\textbf{\opnum\newline \newline \optitle}}}
\vspace{3.0cm}

\begin{tabular}{p{11.0cm}p{2.0cm}}
\hline
Authored by & Date \vspace{1cm}\\

\hline
Approved by & Date \vspace{2cm}\\

\end{tabular}


\newpage

\section*{Introduction}
Metrum Research Group (MetrumRG) intends to qualify the \tfl\ through a series of activities involving functional requirements development, test development, test documentation, testing, and reporting. This document provides the overall qualification plan, and outlines the required elements of the project.

\section*{Definitions}

The following terms may be used throughout the set of qualification documents:

{\bf Envision:} Metworx facilitated platform for leveraging the functionality of shiny-server with the compuational backing of Metworx

{\bf Functional Requirements:}  Processes and functionality that systems (software or hardware) are intended to accomplish.

{\bf Metworx:} A MetrumRG Ruby on Rails web application that provides an interface to launching  AWS computational infrastructure and performing work with the infrastructure in a unified process.

{\bf Qualification:} Successful demonstration, through documentation and testing, of an information system's (or component of the system) functionality, that provides a high degree of assurance that the system will consistently and reproducibly yield a product or result meeting its predetermined specification, expected performance, and/or quality attributes.

{\bf TFL:} Tables, figures, and listings in the typical pharmacometrics workflow

{\bf Validation:} Successful demonstration, through documentation and testing, of a software's intrinsic functionality (correct coding, syntax, start up and run without error).

{\bf Workflow:} The combination of an individually created AWS computational infrastructure and the associated graphical workspace (RStudio, PiranaJS) brought together by Metworx

\newpage

\section*{\tfl\ Overview and Quality Framework}

\tfl\ is a web application, written in R, heavily utilizing the R:::shiny package intended to provide a graphical user interface (GUI) for creating TFL's in the typical pharmacometrics workflow.  The application allows point and click TFL generation, providing output as individual TFL's, a report ready RTF of the TFL's, and an accompanying R script to replicate the analysis.

\section*{\tfl\ Qualification Requirements}

MetrumRG intends to qualify the \tfl\ through a series of activities involving functional requirements development, test development, test documentation, testing, and reporting. The required elements and required documents associated with the project are described in Table \ref{TABLE}. These elements are generally intended to be followed in the order listed. 

\begin{tabular}{|p{5cm}|p{5cm}|p{5cm}|}\hline
\textbf{Element} & \textbf{Description} & \textbf{Output}\\\hline
Qualification Plan & High-level background and outline of the steps to be taken to qualify \tfl, including a list of required documents & Qualification Plan document/section\\
&&\\
% Requirements Specification&A list summarizing the functional requirements&Requirements Specification document/section\\
% &&\\
Requirements Traceability Matrix&Document that maps Requirements to test and documentation references &Requirements Matrix\\
&&\\


Qualification Protocols&Planned stepwise testing sequences to demonstrate that the system capabilities meet the stated requirements&Test protocols\\
&&\\
Qualification Report&Documented results of the qualification testing and a narrative summary of the results including an acceptability assessment&Final report describing outcome qualification testing, problems encountered, and resolution.\\
&&\\
%Training&Documented results of user and organizational administrator training&Training records and training collateral\\
%&&\\
Quality Assurance Release&Official declaration of acceptance and level of release (for training or use in production)&Release document issued upon satisfactory completion of qualification activities\\
&&\\\hline
\end{tabular}\\


\section*{\tfl\ Release and Training}

This qualification plan intends to provide assurance that the application meets its functional requirements and to release \tfl\ for post-qualification training and deployment into a production environment. In order for release into a production environment, training is required for administrators and users.

\section*{Management of \tfl\ Changes}
Once the \tfl\ is qualified, all future \tfl\ modifications are managed through SOP M4 - Change Management.




%%\subsection*{Sections}
%
%%(subsections if needed)
%
%
%\section*{References}
%
%21 CFR Part 11
%
%\newpage
%
%\section*{History}
\end{document}